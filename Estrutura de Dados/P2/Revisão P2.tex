\documentclass[ ]{article}
\usepackage[usenames,dvipsnames]{xcolor} %used for font color
\usepackage[utf8]{inputenc} %useful to type directly diacritic characters
\usepackage[brazil]{babel}
\usepackage{listings}
\usepackage{tcolorbox}
\tcbuselibrary{listings, skins}


\title{Revisão P2 de Estruturas de Dados \\ Dênio Duarte}
\author{Erickson G. Müller}

\begin{document}

\definecolor{codebg}{HTML}{333333}
\definecolor{codetext}{HTML}{ffffff}
\definecolor{codegreen}{HTML}{57db0b}
\definecolor{codeorange}{HTML}{f78e05}
\definecolor{codered}{HTML}{f7052d}
\definecolor{codeshadow}{HTML}{432fa3}

\lstdefinestyle{codestyle}{
	basicstyle =  \color{codetext},
	language = C,
	keywordstyle = \color{codeorange}\bf,
	tabsize = 8
}
\newtcblisting{mylist}{
      enhanced,   %%% needed for shadow
      arc=5mm,
      top=2mm,
      bottom=2mm,
      left=2mm,
      right=12mm,
      boxrule=0pt,
      colback=codebg,
      shadow={1mm}{-1mm}{0mm}{fill=codeshadow,
                      opacity=0.5},             %%% here for shadow  and adjust as you like
      listing only,
      listing options={style=codestyle},
      hbox
}
%\lstset{style=codestyle}

\maketitle

\section{Conteúdos}
\begin{enumerate}
	\item Alocação Dinâmica de Memória
	\item Lista Simplesmente Encadeada
	\item Lista Duplamente Encadeada
	\item Filas e Pilhas
\end{enumerate}
\pagebreak
\section{Alocação Dinâmica de Memória}
	A Alocação Dinâmica de Lista Encadeada na Memória segue os seguintes passos:
	\begin{enumerate}
		\item Alocação de memória
		\item Colocar os valores no espaço alocado e NULLificar as variáveis ponteiras
		\item Encadeamento
	\end{enumerate}
\section{Estudo de Código}
	Desenvolver uma lista encadeada com alocação dinâmica é uma receita de bolo, para isso, iremos analisar cada etapa do código abaixo, que imprime uma sequência de pontos cartesianos.
	
	\begin{mylist}
#include <stdio.h>
#include <stdlib.h>

struct tpoint{
	int x,y;
	struct tpoint *next;
};
typedef struct tpoint tpnt;

int main(){
	tpnt *p, *aux, *first = NULL;
	int i;
	for(i=1;i<=10;i++){
		p = (tpnt*)malloc(sizeof(tpnt));
		
		p->x=i;
		p->y=i+10;
		
		if(first==NULL){
			first = p;
			aux = p;
		}
		else{
			aux->next = p;
			aux = p;
		}
	}
	
	for(aux=first;aux!=NULL;aux=aux->next){
		printf("(%d,%d)\n", aux->x,aux->y);
	}
	
	if(first!=NULL){
		aux=first;
		while(aux->next!=NULL){
			p = aux;
			aux = aux->next;
			free(p);
		}
		free(aux);
		first=NULL;
	}
	return 0;
}
	\end{mylist}

\end{document}