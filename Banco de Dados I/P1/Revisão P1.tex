\documentclass[ ]{article}
\title{Revisão Prova 1\\ Prof. Dênio Duarte}
\author{Erickson G. Müller}

\begin{document}
	\maketitle
	\newpage
	\section{Dados}
		Também chamado de atributo, o dado é a menor característica de um objeto. Os dados têm um domínio e um tipo associados.
	\section{Banco de Dados}
		Coleção de dados relacionados entre si que representam aspectos do minimundo.
		\subsection{Formato Relacional}
			Mais rígido
		\subsection{Formato Semi-Estruturado}
			Organizado com documentos, geralmente JSON
	\section{Dados Armazenados no Disco Rígido}
		\begin{itemize}
			\item Trilha: "Anel" do disco
			\item Setor Geométrico: Seção do disco (formato de pizza)
			\item Setor: Parte da trilha dentro de um setor geométrico.
			\item Bloco: Seção da trilha
		\end{itemize}
		Os dados são posicionados dentro do bloco, a agulha vai girando de trilha em trilha até encontrar o dado correspondente. No disco rígido, o tempo de escrita é o mesmo da leitura.\\
		Embora o SSD  não tenha seek time e rotational time, o tempo de escrita é diferente do tempo de busca.
	\section{Modelos de Dados}
		Os modelos de dados são as formas como os dados são estruturados para os programas acessarem:
		\begin{itemize}
			\item Hierárquico
			\item Redes
			\item Relacional
			\item Orientado a objetos
			\item Objeto-Relacional
			\item Semi-Estruturado
		\end{itemize}
		\subsection{Modelo Relacional}
			Organizado em tabelas, essas tabelas relacionam entre si através das chaves.
	\section{Sistema Gerenciador de Banco de Dados}
		\subsection{Segurança}
			O SGBD vai fazer uma proteção sobre os dados, impedindo que sejam acessados por outros meios que não sejam através do SGBD.
		\subsection{Dicionário}
			Os metadados adicionam flexibilidade, são descrições sobre os dados. Os metadados descrevem como cada objeto é estruturado dentro do banco de dados.
		\subsection{SGBD NoSQL}
			Dados volumosos, não é estritamente estruturado, não é um modelo de dados, mas sim uma classe com diversos modelos (orientado a colunas, orientado a documentos, orientado a grafos...).
	\section{Arquitetura de Três Camadas}
		\begin{itemize}
			\item Visão externa: dados descritos para a aplicação, são apresentadas de forma modelada para cada usuário paramétrico.
			\item Esquema conceitual: dados descritos para os desenvolvedores, vistos conforme a tabela SQL.
			\item Esquema interno:	 dados na forma em que estão armazenaados, visto pelos bytes ocupados pelos dados.
		\end{itemize}
		\subsection{Independência Lógica de Dados}
			Alterar o nível conceitual sem impactar a camada externa.
		\subsection{Independência Física de Dados}
			Alterar a camada interna sem alterar o esquema conceitual.
		\subsection{Integridade Semântica}
			Regras de domínio, integridade, cardinalidade, tamanho de dados, etc. 
	\section{ACID}
		Atomica, Consistente, Isolada, Duradoura.
	\section{BASE}
		Usada no NoSQL: Basic, Available, Soft, Estate.
	\section{Projeto de Banco de Dados}
		Para projetar o nosso SGBD, iremos usar três níveis de abstração: Conceitual, Lógico e Físico.
		\subsection{Modelo Conceitual}
			Após a elicitação de requisitos, será feito um modelo conceitual dos dados. Os nomes dos atributos serão escritos em português.
		\subsection{Modelo Lógico}
			Escolher qual o modelo de dados  será aplicado. Os nomes dos atributos serão escritos em computês.
		\subsection{Modelo Físico}
			
\end{document}