\documentclass[ ]{article}
\title{Revisão Prova 1\\ Prof. Dênio Duarte}
\author{Erickson G. Müller}

\begin{document}
	\maketitle
	\newpage
	\section{Dados}
		Também chamado de atributo, o dado é a menor característica de um objeto. Os dados têm um domínio e um tipo associados.
	\section{Banco de Dados}
		Coleção de dados relacionados entre si que representam aspectos do minimundo.
		\subsection{Formato Relacional}
			Mais rígido
		\subsection{Formato Semi-Estruturado}
			Organizado com documentos, geralmente JSON
	\section{Dados Armazenados no Disco Rígido}
		\begin{itemize}
			\item Trilha: "Anel" do disco
			\item Setor Geométrico: Seção do disco (formato de pizza)
			\item Setor: Parte da trilha dentro de um setor geométrico.
			\item Bloco: Seção da trilha
		\end{itemize}
		Os dados são posicionados dentro do bloco, a agulha vai girando de trilha em trilha até encontrar o dado correspondente. No disco rígido, o tempo de escrita é o mesmo da leitura.\\
		Embora o SSD  não tenha seek time e rotational time, o tempo de escrita é diferente do tempo de busca.
	\section{Modelos de Dados}
		Os modelos de dados são as formas como os dados são estruturados para os programas acessarem:
		\begin{itemize}
			\item Hierárquico
			\item Redes
			\item Relacional
			\item Orientado a objetos
			\item Objeto-Relacional
			\item Semi-Estruturado
		\end{itemize}
		\subsection{Modelo Relacional}
			Organizado em tabelas, essas tabelas relacionam entre si através das chaves.
		\subsection{SGBD NoSQL}
			Dados volumosos, não é estritamente estruturado, não é um modelo de dados, mas sim uma classe com diversos modelos (orientado a colunas, orientado a documentos, orientado a grafos...).
\end{document}