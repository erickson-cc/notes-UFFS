\documentclass[ ]{article}

\usepackage[ ]{amsmath}
\usepackage[ ]{amssymb}
\usepackage[ ]{tikz}

\title{Revisão P1 Geometria Analítica\\Prof. Edson}
\author{Erickson G. Müller}

\begin{document}
	\maketitle
	\newpage
	
	\section*{Conteúdos}
		\begin{itemize}
			\item Vetores
			\item Produto Escalar e Produto Vetorial
			\item Equação de Retas ($\mathbb{R}^2$ e $\mathbb{R}^3$)
			\begin{itemize}
				\item Retas Paralelas
				\item Retas Concorrentes
				\item Retas Reversas
			\end{itemize}
			\item Estudo Relativo de Posições Entre Retas
			\item Equações de Planos
			\begin{itemize}
				\item Planos Paralelos
				\item Planos Concorrentes
			\end{itemize}
			\item Posições Relativas Entre Planos
			\item Distâncias
			\begin{itemize}
				\item entre Pontos
				\item entre Retas
				\item entre Planos
			\end{itemize}
			\item Cônicas
				\begin{itemize}
					\item Elipse
					\item Hipérbole
					\item Parábola
				\end{itemize}
		\end{itemize}
	\section{Retas}
		Sejam $A$ e $B$ dois pontos quaisquer. Vamos escolher $A$ como origem do segmento e $B$ como a extremidade do segmento.\\
		Representação Geométrica:
		\begin{tikzpicture}
			\coordinate (A) at (0,0);
			\coordinate (B) at (3,0);
			\draw[->] (A) -- (B);
			\node at (A) [left] {A};
			\node at (B) [right] {B};
		\end{tikzpicture}\\
		Notação: $AB$\\ \\
		$BA$ é o \textbf{segmento orientado} com origem em $B$ e extremidade em $A$.\\
		Representação Geométrica:
		\begin{tikzpicture}
			\coordinate (A) at (0,0);
			\coordinate (B) at (3,0);
			\draw[->] (B) -- (A);
			\node at (A) [left] {A};
			\node at (B) [right] {B};
		\end{tikzpicture}
		\subsection{Igualdade}
\end{document}