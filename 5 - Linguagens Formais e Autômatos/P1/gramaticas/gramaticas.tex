\documentclass[ ]{article}
\usepackage[ ]{array}
\usepackage[]{amsfonts}
\usepackage[]{amsmath}
\usepackage[ ]{indentfirst}
\usepackage[ ]{graphicx}
\usepackage{booktabs}

\begin{document}

\textbf{1}

	$L(G) = \{ \alpha | \alpha \in (a,b,c)^+$, onde a soma de $a$'s e $c$'s é par se $\alpha$ inicia por $b$, senão $|\alpha|$ é ímpar$\}$
	
	$S ::= a<A>| b<C> | c <A> $ % Inicial
	
	$A ::= a <B> | b<B> | c<B> | \varepsilon$ % inicia por a ou c (|alpha| é ímpar)
	
	$B ::= a <A> | b<A> | c<A>$
	
	$C ::= a<D> | b<C> | c<D> | \varepsilon$ % inicia por b (a+c é ímpar)
	
	$D ::= a<C> | b<D> | c<C>$
	
	
\textbf{2}

	$L(G) = \{\alpha | \alpha \in a^x b^yc^z$ onde $x+z$ é ímpar e $x,y,z>0 \}$
	
	$S::= a <A>$
	
	$A::= a<B> |b<C>$ %x+z ímpar
	
	$B::= a<A>| b<D>$ %x+z par
	
	$C::= b<C> | c<E> $ % x+z ímpar
	
	$D::= b<D> | c<F> $ % x+y par
	
	$E::= c<F>$ % x+y par
	
	$F::= c<E> | \varepsilon$ %x+y ímpar
	
\textbf{3}

	$L(G) = \{ \alpha | \alpha \in (a,b,c)^+$, onde a soma de $a$'s e $c$'s é par se $\alpha$ inicia por $b$, senão $|\alpha|$ é ímpar e $c$ nunca antecede $a\}$
	
	$S::= a<E> | b<A> | c<G>$
	
%inicia por b
	$A::= a<B> | b<A> | c<D> | \varepsilon$ % a+c é par, !c
	
	$B::= a<A> |b<B> | c<C>$ % a+c é ímpar, !c
	
	$C::= b<A>|c<D> | \varepsilon$% a+c é par, c
	
	$D::= b<A>|c<C> $ % a+c é ímpar, c

%inicia por a
	$E::= a<F> | b<F> | c<H> | \varepsilon$ % alpha é ímpar, !c
	
	$F::= a<E> | b<E> | c<G>$ % alpha é par, !c
	
	$G::= b<F> | c<H>| \varepsilon$ % alpha é ímpar, c
	
	$H::= b<E> | c<G>$ %alpha é par, c
	
	\textbf{4}
	
	$L(G) = \{ \alpha | \alpha \in (0$...$9$,$'.'$,$','$,$+$,$-)^+$ onde $\alpha \in \mathbb{R}\}
$

	$digito<= 0 ... 9$
	
	$S'::= +<S> | -<S> | digito<A>$
	
	$S::= digito<A>$
	
	$A::= digito<B> | .<D>| ,<G>| \varepsilon$ % 1 dígito numeral
	
	$B::= digito<C> | .<D> | ,<G> | \varepsilon$ % 2 digitos numeral
	
	$C::=.<D> | ,<G> | \varepsilon$ %3 digitos numeral
	
	% anterior é um ponto (exige a geração de 3 dígitos)
		$D::= digito<E>$
	
		$E::= digito<F>$
	
		$F::=digito<C>$ % obriga a inserção de 1 ponto ou vírgula
	%%%%%%%%%%%%%%%%%
	
	$G::= digito<H>$
	
	$H::= digito<H> | \epsilon$
%	$G::= digito<H> | digito$ %  Essa linha evita a necessidade de criar a regra H pois um símbolo terminal já encerra a sentença
	
	\textbf{Exemplo GLC}
	
	$L(G) =\{ \alpha | \alpha \in a^x c^y$ onde $x>y\}$
	
	$S::= a<S>c | a<A>$
	
	$A::= a<A> | \varepsilon$
	
	$L(G) =\{ \alpha | \alpha \in a^x c^y$ onde $x!=y\}$
	
	$S::= a<S>c | a<A> | <B>c$
	
	$A::= a<A> | \varepsilon$
	
	$B::= <B>c | \varepsilon$
	
	$L(G) =\{ \alpha | \alpha \in a^x b^y c^z$ onde $x!=z$ e $y>0\}$
	
	$S::= a<S>c| a<A> | <B>c$
	
	$A::= a<A> | b<C>$ % quando gerou o b vai para regra C para gerar n quantidade s de b ou encerrar
	
	$B::= <B>c | b<C>$
	
	$C::= b<C> | \varepsilon$
	
	$L(G) =\{ \alpha | \alpha \in a^x b^y c^z$ onde $y=x+z$ e $x,z>0\}$
	% para cada a gerado, obrigatório gerar um b
	% para cada c gerado, obrigatório gerar um b
	
	$S::= <A><B>$
	
	$A::= a<A>b|ab$
	
	$B::=b<B>c|bc$
	
	
	$L(G) =\{ \alpha | \alpha \in (a,b,c)^+$onde o número de $a$'s é igual ao número de $c$'s $\}$ 
	
% não está correta pois a gramática não precisa seguira a ordem a->b->c
%	$S::= a<A>c$
%	
%	$A::= a<A>c |b<B> | \varepsilon$
%	
%	$B::= b<B> | \varepsilon$
	
	$S::= a<> | b<> | c<> | \varepsilon$
	
	$A::=<B><C><B>\textbf{a}<B><C><B>\textbf{c}<C><B> | <B><C><B>\textbf{c}<B><C><B>\textbf{a}<B><C><B> | \varepsilon | <B>$
	
	$B::= b<B> | \varepsilon$
	
	$L(G) = \{ \alpha | \alpha \in (a^{2i+1}b^{i+3} / i>0) \cup (a^{i+4}b^{i+3}/i\geq 0)\}$
	
	$S::= aaa<A>bbbb | aaaa<B>bbb$ % um desvio para cada regra da União
	
	$A::= aa<A>b | \varepsilon$ % contando i com 1 (i>0)
	
	$B::= a<B>b | \varepsilon$ % contando i com 0 (i>=0)
	
	$L(G) = \{ \alpha | \alpha \in ($para, var, = , até, $\{$, $\}$, opl, op, se, então, senão$)^+$ onde $\alpha$ permite estruturas aninhadas de condição e iteração$\}$ %opl operação lógica op operação aritmética %<<<<<<<<<<<<<
	
	$S::= A | B | op$
	
	$A::=$ se opl então {S} $C$
	
	$B::=$ para  $var = var$ até var \{S\}
	
	$C::=$ senão \{S\}  $| \varepsilon$
%	para var opl var { var = var }
%
%	var = var op var
%
%	se var opl var entao var = var senao var = var
%
%	var = var


	
	
	\newpage

%\textbf{4}
%	 Minha tentativa antes da correção do professor
%	$L(G) = \{ \alpha | \alpha \in (0...9, +,-,'.',',')^+$e $d \in \mathbb{R}\}$
%
%	$NUMEROS::= 0 | 1 | 2 | 3 | 4  | 5 | 6 | 7 | 8 | 9 $
%	
%	$POSITIVOS::=  1 | 2 | 3 | 4  | 5 | 6 | 7 | 8 | 9 $
%	
%	$SINAIS ::= + | -$
%	
%	$PONTO ::= .$
%	
%	$VIRGULA ::= ,$
%	
%	$ZERO::= 0$
%	
%	$S::= SINAIS <A> | ZERO<> | POSITIVOS<A>$
%	% Faltou diferenciaro  número de dígitos para procurando ponto quando começa com positivo e quando começa com sinal. Portanto os dois não podem direcionar para A em S.
%	% vou comentar essa resposta para prestar atenção na explicação do professor
%	
%% comecou com sinal
%	$A::= ZERO<B> | POSITIVOS<Q0>|VIRGULA<DECIMAL>$
%	
%	$Q0::= NUMEROS<PP>|PONTO <Q1>| VIRGULA<DECIMAL>$
%	
%	$Q1::= NUMEROS<Q2>$
%
%	$Q2::= NUMEROS<Q3>$	
%
%	$Q3::= NUMEROS<Q4>$
%	
%	$Q4::= PONTO <Q1> | VIRGULA<DECIMAL>$	
%	
%	$PP ::= NUMEROS<PP2>|PONTO<Q1>| VIRGULA<DECIMAL>$
%	
%	$PP2 ::= PONTO<Q1> | VIRGULA<DECIMAL>$
%	$B::= VIRGULA <DECIMAL>$
%	
%	$DECIMAL::= NUMEROS<DECIMAL>$

	\section*{Lista 1 - Gramáticas Regulares}
	
	\textbf{a}
	
	$L(G) = \{x | x \in (a,b)^*$onde o número de $b$'s é par$\}$
	
	$S::= a<B> | b<A> | \varepsilon$
	
	$B::= a<B> | b<A> | \varepsilon$ % b é par
	
	$A::= a<A> | b <B> $ % b é ímpar
	
	\textbf{b}
	
	$L(G) = \{x | x \in (a,b)^*$onde o número de $b$'s é par$\}$
	
	$S::= a<A> | b<B> $ % b é par
	
	$A::= a<A> | b<B>$ % b é par
	
	$B::= a<B>| b<A> | \varepsilon$ % b é ímpar
	
	\textbf{c}
	
	$L(G) = \{ x|x \in (a,b,c)^*$onde ocorra pelo menos dois padrões $'ac'\}$
	
	$S::= a<B> | b<A> | c<A>$
	
	% primeira repetição
	$A::= a<B> | b<A> | c<A>$ % anterior não é a 
	
	$B::= a<B> | b<A> | c<C>$ % anterior é a
	
	% segunda repetição
	$C::= a<D> | b<C> | c<C>$ % anterior não é a
	
	$D::= a<D> | b<C> | c<E>$ % anterior é a
	
	% já cumpriu o requisito
	$E::= a<E> | b<E> | c<E> | \varepsilon$
	
	\textbf{d}
	
	$L(G) = \{ x|x \in (a,b,c)^*$onde ocorra pelo menos um padrão $'abc'\}$
	
	$S::= a<B> | b<A> | c<A>$ 

	$A::= a<B> | b<A> | c<A>$ %anterior não é a
	
	$B::= a<B> | b<C> | c<A>$ %anterior é a
	
	$C::= a<B> | b<A> | c<D>$ % anterior é ab
	
	$D::= a<D> | b<D> | c<D> | \varepsilon$ % pode encerrar
	
	\textbf{e}
	
	$L(G) = \{ x|x \in (0,1)^*$onde o número de $1$'s é múltiplo de $3\}$
	
	$S::= 0<S> | 1<A> | \varepsilon$% K mod 3 = 0	
	
	$A::= 0<A> | 1<B>$ % K mod 3 = 1
	
	$B::= 0<B> | 1<S> $ % K mod 3 = 2
	
	\textbf{f}
	
	$L(G) = \{ x | x \in (a,b,c,d)^+$onde a soma de $a$'s e $c$'s é ímpar se $x$ começa com $a$ ou a soma de $a$'s e $d$'s é par se $x$ começa por $b$; se $x$ inicia por $c$ ou $d$ não existe restrição$\}$
	
	$S::= a<A> | b<C> | c<E> | d<E> $ % inicio
	
	$A::= a<B> | b<A> | c<B> | d<A> | \varepsilon$ % inicia por a & a+c é ímpar
	
	$B::= a<A> | b<B> | c<A> | d<B>$ % inicia por a & a+c é par
	
	$C::= a<D> | b<C> | c<C> | d<D> | \varepsilon$ % inicia por b & a+d é par

	$D::= a<C> | b<D> | c<D> | d<C>$ % inicia por b & a+d é ímpar	
	
	$E::= a<E> | b<E> | c<E> | d<E> | \varepsilon$
	\section*{Lista 2 - Autômatos Finitos}
		\textbf{a}
		
		$S::= 0<S> | 1<S> | 0<A> | 0<C> | 1<B>$
		
		$A::= 0<A> | 0<C> | 0$
		
		$B::= 1<B> | 1$
		
		$C::= 0<C> | 0<A> | 0$
		
		\includegraphics[scale=0.5]{../images/automato-exemplo-1.png}
		
		\textbf{b}
		
		$S::= a<A> | a<C> | b<B> | b<C>$
		
		$A::- a<F> | a$
		
		$B::= b<F> | b$
		
		$C::= a<A> | a<C> | b<B> | b<C>$
		
		$F::= a<F> | b<F> | a | b$
		
		\includegraphics[scale=0.5]{../images/automato-exemplo-2.png}
		\newpage
	\section*{Lista de exercícios nova - Erickson G. Müller}
		\subsection{4 - autômatos finitos}

		\textbf{a}
		
		$S::= 0S | 1S | 0A | 0C | 1B$
		
		$A::= 0A | 0C | 0$
		
		$B::= 1B | 1$
		
		$C::= 0C | 0A | 0$
		
		
		\includegraphics[scale=0.3]{images/4-a-afnd.png}
		
\begin{center}
    % Minipage para o AFND
    \begin{minipage}[t]{0.48\textwidth} % t alinha pelo topo
        \centering % Centraliza o conteúdo dentro da minipage
        \textbf{AFND}\\ % Título da tabela
        \vspace{0.2cm} % Espaço entre o título e a tabela
        \begin{tabular}{|c|c|c|} % Colunas com linhas verticais
            \toprule % Linha superior do booktabs
            & 0 & 1 \\
            \midrule % Linha intermediária do booktabs (espaçamento reduzido)
            $\to$S & S,A,C & S,B\\
            A & A,C,F & -\\
            B & - &  B, F\\
            C & A, C, F & -\\
            *F & - & - \\
            \bottomrule % Linha inferior do booktabs
        \end{tabular}
    \end{minipage}%
    \hfill % Espaço horizontal entre as minipages
    % Minipage para o AFD
    \begin{minipage}[t]{0.48\textwidth} % t alinha pelo topo
        \centering % Centraliza o conteúdo dentro da minipage
        \textbf{AFD}\\ % Título da tabela
        \vspace{0.2cm} % Espaço entre o título e a tabela
        \begin{tabular}{|l|c|c|} % Colunas com linhas verticais. 'l' para a primeira coluna para alinhamento da seta/asterisco
            \toprule % Linha superior do booktabs
            & 0 & 1 \\
            \midrule % Linha intermediária do booktabs (espaçamento reduzido)
            $\to$S & $[S,A,C]$ & $[S,B]$\\ % Corrigido: sem o * aqui para o estado inicial
            $[S,A,C]$ & $*[S,A,C,F]$ & $[S,B]$ \\
            $*[S,A,C,F]$ & $*[S,A,C,F]$ & $[S,B]$ \\
            $[S,B]$ & $[S,A,C]$ & $*[S,B,F]$\\
            $*[S,B,F]$ & $[S,A,C]$ & $*[S,B,F]$\\
            \bottomrule % Linha inferior do booktabs
        \end{tabular}
    \end{minipage}
\end{center}
		
%%%%%%%%%%%%%%%%%%%%%%%%%%%%%%%%%%%%%%%%%%%%%%%%%%%%%%%%%%%%%%%%%%%%%%%%%%%%%%%%%%%%%%
		\newpage
		\textbf{b}
		
		$S ::= aA | aC | bB | bC$
		
		$A ::= aF | a$
		
		$B ::= bF | b$
		
		$C ::= aA | aC | bB | bC$
		
		$F ::= aF | bF | a | b$
		
		% includegraphics
		
\begin{center}
    % Minipage para o AFND
    \begin{minipage}[t]{0.48\textwidth} % t alinha pelo topo
        \centering % Centraliza o conteúdo dentro da minipage
        \textbf{AFND}\\ % Título da tabela
        \vspace{0.2cm} % Espaço entre o título e a tabela
        \begin{tabular}{|c|c|c|} % Colunas com linhas verticais
            \toprule % Linha superior do booktabs
            & a & b \\
            \midrule % Linha intermediária do booktabs (espaçamento reduzido)
            $\to S$ & $A,C$ & $B,C$\\
            $A$ & $F,K$ & - \\
            $B$ & - & $F,K$\\
            $C$ & $A,C$ & $B,C$\\
            $F$ & $F,K$ & $F,K$\\
            $*K$ & - & - \\
            \bottomrule % Linha inferior do booktabs
        \end{tabular}
    \end{minipage}%
    \hfill % Espaço horizontal entre as minipages
    % Minipage para o AFD
    \begin{minipage}[t]{0.48\textwidth} % t alinha pelo topo
        \centering % Centraliza o conteúdo dentro da minipage
        \textbf{AFD}\\ % Título da tabela
        \vspace{0.2cm} % Espaço entre o título e a tabela
        \begin{tabular}{|l|c|c|} % Colunas com linhas verticais. 'l' para a primeira coluna para alinhamento da seta/asterisco
            \toprule % Linha superior do booktabs
            & a & b \\
            \midrule % Linha intermediária do booktabs (espaçamento reduzido)
			$\to S$ & $[A,C]$ & $[B,C]$\\
			$[A,C]$ & $*[F,K]$ & $[B,C]$\\
			$[B,C]$ & $[A,C]$ & $*[B,C,F,K]$\\
			$*[F,K]$ & $*[F,K]$ & $*[F,K]$\\
			$*[B,C,F,K]$ & $*[A,C,F,K]$ & $*[B,C,F,K]$\\
			$*[A,C,F,K]$ & $*[A,C,F,K]$ & $*[B,C,F,K]$\\
            \bottomrule % Linha inferior do booktabs
        \end{tabular}
    \end{minipage}
\end{center}
%%%%%%%%%%%%%%%%%%%%%%%%%%%%%%%%%%%%%%%%%%%%%%%%%%%%%%%%%%%%%%%%%%%%%%%%%%%%%%%%%%%%%%
		\newpage
		\textbf{c}
		
		$S ::= aA | bB$
		
		$A ::= aS | aC | a$
		
		$B ::= bS | bD | b$
		
		$C ::= aB$
		
		$D ::= bA$
		
\begin{center}
    % Minipage para o AFND
    \begin{minipage}[t]{0.48\textwidth} % t alinha pelo topo
        \centering % Centraliza o conteúdo dentro da minipage
        \textbf{AFND}\\ % Título da tabela
        \vspace{0.2cm} % Espaço entre o título e a tabela
        \begin{tabular}{|c|c|c|} % Colunas com linhas verticais
            \toprule % Linha superior do booktabs
            & a & b \\
            \midrule % Linha intermediária do booktabs (espaçamento reduzido)
		$\to S$ & $A$ & $B$\\
		$A$ & $S,C,F$ & -\\
		$B$ & - & $S,D,F$\\
		$C$ & $B$ & -\\
		$D$ & - & $A$\\
		$*F$ & - & -\\
            \bottomrule % Linha inferior do booktabs
        \end{tabular}
    \end{minipage}%
    \hfill % Espaço horizontal entre as minipages
    % Minipage para o AFD
    \begin{minipage}[t]{0.48\textwidth} % t alinha pelo topo
        \centering % Centraliza o conteúdo dentro da minipage
        \textbf{AFD}\\ % Título da tabela
        \vspace{0.2cm} % Espaço entre o título e a tabela
        \begin{tabular}{|l|c|c|} % Colunas com linhas verticais. 'l' para a primeira coluna para alinhamento da seta/asterisco
            \toprule % Linha superior do booktabs
            & a & b \\
            \midrule % Linha intermediária do booktabs (espaçamento reduzido)
		$\to S$ & $A$ & $B$\\
		$A$ & $*[S,C,F]$ & -\\
		$*[S,C,F]$ & $[A,B]$ & $B$\\
		$B$ & - & $*[S,D,F]$\\
		$[A,B]$ & $*[S,C,F]$ & $*[S,D,F]$\\
		$*[S,D,F]$ & $A$ & $[A,B]$\\
            \bottomrule % Linha inferior do booktabs
        \end{tabular}
    \end{minipage}
\end{center}

%%%%%%%%%%%%%%%%%%%%%%%%%%%%%%%%%%%%%%%%%%%%%%%%%%%%%%%%%%%%%%%%%%%%%%%%%%%%%%%%%%%%%%
		\newpage
		\textbf{d} % Corrigida pelo professor
		
		$S::= 0<B> | 1<A> | 1 | \varepsilon$
		
		$A::= 0<B> | \varepsilon$
		
		$B::= 0<C> | 0 | 1<D>$
		
		$C::= 0<B> | 1<A> | 1$
		
		$D::= 1<C> | 1$
		
		\begin{center}
		AFND\\
		
		\begin{tabular}{ c c c }
  & 0 & 1 \\ 
 $\to$S & B & A,X \\  
 *X & - & - \\
 *A & B & - \\
 B & C, X & D\\
 C& B & A,X\\
 D & - & C, X
		\end{tabular}
		\end{center}
		
		\begin{center}
		AFD\\
		
		\begin{tabular}{ c c c }
   & 0 & 1 \\ 
 $\to$*S & B & [AX] \\  
 B & [CX] & D\\
 *[AX] & B & - \\
 *[CX] & B & [AX]\\
 D & - & [CX]
 		\end{tabular}
		\end{center}
%%%%%%%%%%%%%%%%%%%%%%%%%%%%%%%%%%%%%%%%%%%%%%%%%%%%%%%%%%%%%%%%%%%%%%%%%%%%%%%%%%%%%%
	\newpage
		\textbf{e}
		
		$S ::= aA | bB | a$
		
		$A ::= aS | bC$
		
		$B ::= aC | bS$
		
		$C ::= aB | bA | b$
		
\begin{center}
    % Minipage para o AFND
    \begin{minipage}[t]{0.48\textwidth} % t alinha pelo topo
        \centering % Centraliza o conteúdo dentro da minipage
        \textbf{AFND}\\ % Título da tabela
        \vspace{0.2cm} % Espaço entre o título e a tabela
        \begin{tabular}{|c|c|c|} % Colunas com linhas verticais
            \toprule % Linha superior do booktabs
            & a & b \\
            \midrule % Linha intermediária do booktabs (espaçamento reduzido)
		$\to S$ & $A,F$ & $B$\\
		$A$ & $S$ & $C$\\
		$B$ & $C$ & $S$\\
		$C$ & $B$ & $A,F$\\
		$*F$ & $-$ & $-$\\
            \bottomrule % Linha inferior do booktabs
        \end{tabular}
    \end{minipage}%
    \hfill % Espaço horizontal entre as minipages
    % Minipage para o AFD
    \begin{minipage}[t]{0.48\textwidth} % t alinha pelo topo
        \centering % Centraliza o conteúdo dentro da minipage
        \textbf{AFD}\\ % Título da tabela
        \vspace{0.2cm} % Espaço entre o título e a tabela
        \begin{tabular}{|l|c|c|} % Colunas com linhas verticais. 'l' para a primeira coluna para alinhamento da seta/asterisco
            \toprule % Linha superior do booktabs
            & a & b \\
            \midrule % Linha intermediária do booktabs (espaçamento reduzido)
		$\to S$ & $*[A,F]$ & $B$\\
		$*[A,F]$ & $S$ & $C$\\
		$B$ & $C$ & $S$\\
		$C$ & $B$ & $*[A,F]$\\
		$*F$ & - & -\\
            \bottomrule % Linha inferior do booktabs
        \end{tabular}
    \end{minipage}
\end{center}
%%%%%%%%%%%%%%%%%%%%%%%%%%%%%%%%%%%%%%%%%%%%%%%%%%%%%%%%%%%%%%%%%%%%%%%%%%%%%%%%%%%%%%
		\textbf{f}
		
		$S ::= aA | bB | b | cS | c | \varepsilon$
		
		$A ::= aS | a | bC | cA$

		$B ::= aA | cB | cS | c$

		$C ::= aS | a | cA | cC$
		
\begin{center}
    % Minipage para o AFND
    \begin{minipage}[t]{0.48\textwidth} % t alinha pelo topo
        \centering % Centraliza o conteúdo dentro da minipage
        \textbf{AFND}\\ % Título da tabela
        \vspace{0.2cm} % Espaço entre o título e a tabela
        \begin{tabular}{|c|c|c|c|} % Colunas com linhas verticais
            \toprule % Linha superior do booktabs
            & a & b & c\\
            \midrule % Linha intermediária do booktabs (espaçamento reduzido)
		$\to *S$ & $A$ & $B,F$ & $S,F$\\
		$A$ & $S,F$& $C$ & $A$\\
		$B$ & $A$ & - & $B,S,F$\\
		$C$ & $S,F$ & - &$A,C$\\
		$F$ & - & - & -\\
            \bottomrule % Linha inferior do booktabs
        \end{tabular}
    \end{minipage}%
    \hfill % Espaço horizontal entre as minipages
    % Minipage para o AFD
    \begin{minipage}[t]{0.48\textwidth} % t alinha pelo topo
        \centering % Centraliza o conteúdo dentro da minipage
        \textbf{AFD}\\ % Título da tabela
        \vspace{0.2cm} % Espaço entre o título e a tabela
        \begin{tabular}{|l|c|c|c|} % Colunas com linhas verticais. 'l' para a primeira coluna para alinhamento da seta/asterisco
            \toprule % Linha superior do booktabs
            & a & b & c\\
            \midrule % Linha intermediária do booktabs (espaçamento reduzido)
		$\to *S$ & $A$ & $[B,F]$ & $[S,F]$\\
		$A$ & $[S,F]$ & $C$ & $ A$\\
		$[B,F]$ & $A$ & $-$ & $[B,S,F]$\\
		$[S,F]$ &  $A$ & $[B,F]$ & $[S,F]$\\
		$C$ & $[S,F]$ & - & $[A,C]$\\
		$[B,S,F]$ & $A$ & $[B,F]$ & $[B,S,F]$\\
		$[A,C]$ & $[S,F]$ & $C$ & $[A,C]$\\
            \bottomrule % Linha inferior do booktabs
        \end{tabular}
    \end{minipage}
\end{center}

	\section*{5 Classe de equivalência}
	
		\textbf{a}
		
		$ S::= aS | aB | bS$
		
		$B::= aC$
		
		$C::= b$
		\begin{minipage}[t]{0.48\textwidth} % t alinha pelo topo
        \centering % Centraliza o conteúdo dentro da minipage
        \textbf{}\\ % Título da tabela
        \vspace{0.2cm} % Espaço entre o título e a tabela
        \begin{tabular}{|l|c|c|c|} % Colunas com linhas verticais. 'l' para a primeira coluna para alinhamento da seta/asterisco
            \toprule % Linha superior do booktabs
             & a & b \\
            \midrule % Linha intermediária do booktabs (espaçamento reduzido)
		S & $S,B$ & $S$\\
		$B$ & $C$ & \\
		$C$ & & $D$\\
		$*D$ & & \\
            \bottomrule % Linha inferior do booktabs
        \end{tabular}
    \end{minipage}
    
    		\begin{minipage}[t]{0.48\textwidth} % t alinha pelo topo
        \centering % Centraliza o conteúdo dentro da minipage
        \textbf{AFD}\\ % Título da tabela
        \vspace{0.2cm} % Espaço entre o título e a tabela
        \begin{tabular}{|l|c|c|c|} % Colunas com linhas verticais. 'l' para a primeira coluna para alinhamento da seta/asterisco
            \toprule % Linha superior do booktabs
             & a & b \\
            \midrule % Linha intermediária do booktabs (espaçamento reduzido)
		S & $[S,B]$ & $S$\\
		$[S,B]$ & $[S,B,C]$ & $S$\\
		$[S,B,C]$ & $[S,B,C]$ & $[S,D]$\\
		$*[S,D]$ &$[S,B]$& $S$\\
            \bottomrule % Linha inferior do booktabs
        \end{tabular}
    \end{minipage}
		
		
	\section{7 Produções epsilon}
		\textbf{b} \textit{diferente da lista}
		
		Marcar produções que ó first é o Epsilon
		
		$S::= \textbf{A}\textbf{C} | aB\textbf{C} | b\textbf{C}$
			% Agora incluir as produções caso substiuir por Epsilon
			$| \textit{C}| \textit{A} | \textit{aB} | \textit{b}$ 
		
		$\textbf{A}::= a\textbf{A}a | \varepsilon \textit{-}$ $| \textit{aa}$ % eliminar o epsilon
		
		$B::= B |bB$
		
		$\textbf{C}::= \textbf{C}B | c\textbf{C} | \varepsilon \textit{-}$ $| \textit{B} | \textit{c}$
	\section*{9 Fatoração de gramáticas}
	
		\textbf{d}
		
		$S::= aBd | acD|bc$
		
		$B::= bDc | bCd | ad$
		
		$D::= cdD | cdB$ \textit{Professor alterou a segunda regra}
		
		$C:: cbB | adD$
		
		
		$S::= aS'| bC$
		
		$S'::= Bd|cD$
		
		$B::= bB' | ad$
		
		$B' ::= Dc|Cd$
		
		$D::= cdD'$

		$D'::= D | B$
		
		$B' ::= cdD'c| cbBd | adDd$
		
		$B'::=CB''|adDd$
		
		$B'' ::= dD'c |bBd$
	\section*{10 Eliminar recursão à esquerda}
		
		\textbf{d}
		
		$S::= Aab|Bc|\overbrace{S} cAb$
		
		$A::= SAc | BaA| ab$
		
		$B::= Ac|aBb|ab$
		
		$S::= AabS'| BcS'$
		
		$S'::= cAbS' | \varepsilon$
		
		$A::=AabS'Ac | BcS'Ac | BaA|ab$ % recursão indireta quando A começa com S e S começa com A
		
		$A::= BcS'AcA' | BaAA'|abA'$
		
		$A'::= abS'AcA'|\varepsilon$
		
		$B::= BcS'AcA'c | BaAA'c| abA'c| aBb|ab$
		
		$B::= abA'cB' | aBbB' |abB'$
		
		$B'::= cS'AcA'cB'| aAA'cB' | \varepsilon$
		
		Lembrar de remover as produções que foram subistituídas
	\section*{11 First e Follow}
	o follow é o first do símbolo que vem logo em seguida
	
	\textbf{a} \textit{feita pelo professor}
	
	$S::= Syx | Bz | CAw|AB$
	
	$A::= aCB | Byb|bC$
	
	$B::= cAd| Byd | aB|\varepsilon$
	
	$C::= zBd |wCc|ABy|\varepsilon$
	
	    \begin{minipage}[t]{0.48\textwidth} % t alinha pelo topo
        \centering % Centraliza o conteúdo dentro da minipage
        \textbf{}\\ % Título da tabela
        \vspace{0.2cm} % Espaço entre o título e a tabela
        \begin{tabular}{|l|c|c|c|} % Colunas com linhas verticais. 'l' para a primeira coluna para alinhamento da seta/asterisco
            \toprule % Linha superior do booktabs
             & First & Follow\\
            \midrule % Linha intermediária do booktabs (espaçamento reduzido)
		 S & $c,a,z,w,b,\varepsilon, y$ &\$ $,y$ \\
		 A & $a,b,c, y$ & $\underbrace{w},\underbrace{c,a,y},\underbrace{d}$ $\$ $\\%segunda passada F(A) = F(A)+F(S)
		 B & $c,a,\varepsilon, y$ & $z,y,d$ $\$,w,c,a $ \\ % segunda passada F(B) = F(B)+F(S)
		 C & $z,w,a,b, \varepsilon, c,y$ & $a,b,c,y$ $w,d,\$ $\\ % segunda passada
            \bottomrule % Linha inferior do booktabs
        \end{tabular}
    \end{minipage}
	
\end{document}
