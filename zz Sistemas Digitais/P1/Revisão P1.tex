\documentclass[•]{article}
\title{Revisão P1\\Prof. Geomar}
\date{26 de Setembro de 2024}
\author{Erickson G. Müller}

\begin{document}
	\maketitle
	\section*{Conteúdo}
		\begin{enumerate}
			\item Circuitos Combinacionais
			\item Circuito Contador Binário
			\item Circuito Contador de Gray
			\item Linguagem de Descrição de Hardware VHDL
		\end{enumerate}
	\newpage
	
	\section{Sistema Digital}
		Um aparato dotado de conjuntos finitos de \textbf{entradas} e \textbf{saídas} e capaz de processar informação representada sob forma \textbf{numérica}.
	\section{Circuitos Combinacionais}
		Não armazenam dados, exemplo: Circuito Somador.
	\section{Circuitos Sequenciais}
		Armazenam dados, exemplo: Circuito Contador.
	\section{Níveis de Abstração}
		Em ordem decrescente:
		\begin{enumerate}
			\item Nível de Sistema: CPU, ASIP, ASIC, barramentos, memórias, software embarcado.
			\item Nível RT (Transferência entre Registradores): Unidades funcionais (somadores, subtratores, multiplicadores), Rede de interconexão (fios, multiplexadores, decodificadores, barramentos, buffers tri-state), Registradores e blocos de memória RAM, ROM.
			\item Nível Lógico: portas lógicas, latches e flip-flops.
			\item Nível de Circuito Elétrico: transistores, resistores, capacitores, indutores e fios.
			\item Nível de Transistor.
		\end{enumerate}
	\section{Circuitos Contadores}
		\textbf{Módulo do contador:} é o número de estados que o contador passa antes de retornar ao 0, é o número de valores que o contador pode ter, não o valor máximo. Para um circuito formado por $n$ flip-flops, teremos um contador de módulo máximo $2^n$.\\
		\textbf{Contador Assíncrono:} Não tem a entrada de clock comum a todos os flip-flops, o sinal de clock não é aplicado a todos mas sim ao flip-flop que representa o bit de menor valor. Para montar um contador assíncrono colocamos o J e K na fonte, ligamos o clock de cada flip-flop à saída negada do flip-flop anterior e colocamos uma porta AND 111 para determinar quando o contador vai resetar.\\
		\textbf{Contador Síncrono:} O sinal de clock é comum a todos os flip-flops do circuito. Para fazer um circuito contador síncrono, faz-se a tabela dos estados atuais, os estados seguintes, e compara-se para saber quais os valores das entradas JK dos flip-flops. Em seguida monta-se os mapas de karnaugh, e tiramos as expressões simplificadas de cada flip-flop em relação às saídas dos outros flip-flops.
	\section{VHDL}
		Lembrar que "$<$=" se manifesta no tempo seguinte e ":=" se manifesta instantaneamente. E acredito que quando usamos o := numa variável, não precisamos criar um signal antes do process mas sim uma variable após o process. Na dúvida, não colocar nem um nem outro.\\
		\subsection{Representação de um process(A,B)}
			\begin{enumerate}
				\item \textbf{Algorítmica}:\\
					if(B$<$A) then s$<=$ '1';\\
					else s$<=$ '0';\\
					end if;
				
				\item \textbf{Fluxo de Dados}:\\
					s$<=$ '1' when B$<$A else '0';
				
				\item \textbf{Estrutural}:\\
					Muito mais complicado, tem signal, tem component, tem port map.\\
					U1: xor2 port map (A,B,L1);\\
					U2: and2 port map (A,L!,S);
			\end{enumerate}
\end{document}
