\documentclass[]{article}
\usepackage[ ]{indentfirst}
\usepackage{tikz}
\usetikzlibrary{automata, positioning}
\usepackage{booktabs}
\usepackage[ ]{array}

\usepackage[ ]{graphicx}
\title{Revisão P2\\Linguagens Formais e Autômatos}
\author{Erickson G. Müller}
\date{}

\begin{document}
\maketitle
\section*{Conteúdos}

\section{Minimização de Autômatos Finitos}
Um autômato finito é mínimo se não possui \textbf{estados inacessíveis, mortos e equivalentes}.\\
Considerando o seguinte AFND, vamos reconhecer esses estados\\
\begin{minipage}[t]{0.43\textwidth}
\centering
\begin{tabular}{|c|c|c|} % Colunas com linhas verticais
            \toprule % Linha superior do booktabs
            & a & b \\
            \midrule % Linha intermediária do booktabs (espaçamento reduzido)
		$\to *A $ & $G$ & $B$\\
		$B$ & $F$ & $E$\\
		$C$ & $C$ & $G$\\
		$*D$ & $A$ & $H$\\
		$E$ & $E$ & $A$\\
		$F$ & $B$ & $C$\\
		$*G$ & $G$ & $F$\\
		$H$ & $H$ & $D$\\
            \bottomrule % Linha inferior do booktabs
        \end{tabular}
\end{minipage} 
\hfill
\begin{minipage}{0.43\textwidth}   
$A\to \{ G,B,F,E,C,A$ \textit{acess}\\
$B\to \{F,E,C,A,B,G$\\
$C\to \{C,G,B,F,E,A$\\
$D \to \{A,H,G,B,F,E,C,D$\\
$E \to \{ E,A,G,B,F,C$\\
$F\to \{ B,C,F,E,A,G$\\
$G\to \{ G,F,B,C,E,A$\\
$H \to \{ H,D,A,C,G,B,F,E$\\
\end{minipage}
    
O algoritmo para encontrar os estados inacessíveis começa com uma lista-chave de todos os estados seguidos dos estados que são possíveis de ser alcançado através daquele. Exemplo $A \to \{ G, B$; Todos os estados que estão à direita compartilham o estados que se podem acessar com o estado à esquerda.

A segunda parte do algoritmo consiste em adicionar esses estados que podem ser acessados em duas transições através dos estados que podem ser acessados diretamente. Exemplo $A \to \{ G, B,F, E$, e assim em ciclo até completar todos os estados.

Por fim, temos que os estados inalcançáveis são $D$ e $H$.

Os estados que são mortos são aqueles que não alcançam, por nenhum fecho transitivo, um estado final. Portanto são os estados

Os estados são equivalentes quando, para cada símbolo, levam ao mesmo tipo de estados. Por exemplo, o $A$ e o $G$ são equivalentes pois $G$ e $G$ são estados finais e $B$ e $F$ são estados não finais.

\begin{minipage}[t]{0.43\textwidth}
\begin{tabular}{|c|c|} % Colunas com linhas verticais
            \toprule % Linha superior do booktabs
            $F$ & $K-F $\\
            \midrule % Linha intermediária do booktabs (espaçamento reduzido)
		$A,G$ & $B,C,E,F$\\
		$[A,G]$ & $[B,C],[C,E]$\\
            \bottomrule % Linha inferior do booktabs
        \end{tabular}
\end{minipage} 
\hfill
\begin{minipage}[t]{0.43\textwidth}
\begin{tabular}{|c|c|c|} % Colunas com linhas verticais
            \toprule % Linha superior do booktabs
             & $a$ & $b$\\
            \midrule % Linha intermediária do booktabs (espaçamento reduzido)
		$\to *[AG]$ & $[AG]$ & $[BF]$\\
		$[BF]$ & $[BF]$ & $[CF]$\\
		$[CE]$ & $[CE]$ & $[AG]$\\
		
            \bottomrule % Linha inferior do booktabs
        \end{tabular}
\end{minipage} 

	\subsection{Exercícios de minimização}
\end{document}
