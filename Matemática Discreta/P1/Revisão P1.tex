\documentclass[•]{article}
\title{Revisão Prova 1\\Matemática Discreta\\Neri}
\author{Erickson G. Müller}

\begin{document}
	\maketitle
	\section{Conteúdos}
		\begin{enumerate}
			\item Proposições logicamente equivalentes
			\item Lógica proposicional
			\item Argumentos válidos, argumentos verbais
			\item Regras de inferência
			\item Lógica de predicados
			\item Quantificadores universal e existencial
			\item Regras de inferência para quantificadores
			\item Técnicas de demonstração: direta, contraposição, exaustão e absurdo
			\item Teoria dos conjuntos, subconjuntos
			\item Álgebra dos conjuntos
			\item Relações: binárias, equivalência
			\item Partições
			\item Funções: domínio e imagem
			\item Funções: injetora, sobrejetora, bijetora
			\item Composição de funções
			\item Função inversa
		\end{enumerate}
	\newpage
	A matemática pode ser dividida em dois \textbf{domínios}: o \textit{contínuo} e o \textit{discreto}. A matemática contínua estuda conceitos infinitos em seu objetivo, utilizando o sistema de números reais. A matemática discreta utiliza um domínio de números que não estão conectados da mesma forma que os números reais. É uma comparação semelhante à diferença entre o sinal analógico e o digital.\\
	
	A matemática discreta exige do aluno que sejam desenvolvidas demonstrações (provas), para isso existem diversos \textbf{esquemas de provas} que se aplicam a cada caso. O autor do livro recomenda elaborar as provas escrevendo a primeira e a última sentença, e ir desenvolvendo em direção ao meio até que ambas se encontrem.\\
	
	

\end{document}