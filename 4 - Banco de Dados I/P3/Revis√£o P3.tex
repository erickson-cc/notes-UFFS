\documentclass[ ]{article}
\title{Revisão Prova 3\\ Prof. Dênio Duarte}
\author{Erickson G. Müller}

\begin{document}
	\maketitle
	\newpage
	\section{Conteúdos}
		\begin{itemize}
			\item Álgebra Relacional
			\item Operadores Unários
			\item Operadores Binários
			\item Álgebra Estendida
			\item Dependências Funcionais
			\item Primeira Forma Normal
			\item Segunda Forma Normal
			\item Normalização de Boyce-Codd
		\end{itemize}
	Um operador de uma álgebra retorna uma relação\\
	Não existe NULL na álgebra relacional
	\section{Operadores Primitivos}
		\begin{itemize}
			\item Projeção (unário)
			\item Seleção (unário)
			\item Cartesiano (binário)
			\item Join (binário) [left, right, full]
			\item União (binário)
			\item Intersecção (binário)
			\item Diferença (binário)
			\item Divisão (binário)
			\item Agregação (grupo)
		\end{itemize}
		Precisa-se fazer primeiro a seleção e depois a projeção, na maioria dos casos.
		$$\pi_{(nome, pop)}(\omega_{(area>500)}(cidade)$$
	\section{Divisão}
		A divisão vai dividir uma tabela para outra. Para tal, os atributos de uma tabela a ser dividida deve estar contida na tabela a dividir.
\end{document}
