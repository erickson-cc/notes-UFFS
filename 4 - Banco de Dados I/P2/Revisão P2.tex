\documentclass[ ]{article}
\title{Revisão Prova 2\\ Prof. Dênio Duarte}
\author{Erickson G. Müller}

\begin{document}
\maketitle
\newpage
\section{Projeto de Banco de Dados}
	Após a elicitação de requisitos, o projeto de banco de dados possui 3 fases:
	\begin{enumerate}
		\item Modelagem Conceitual
		\item Modelagem Lógica
		\item Modelagem Física
	\end{enumerate}
	
	\subsection{Modelo Conceitual}
		Independe do modelo de dados do SGBD.
	\subsection{Modelo Lógico}
		Escolhe-se um modelo de dados.
		\subsubsection{Modelo Relacional}
			Os dados são organizados em forma de tabelas (Relações).
			\begin{itemize}
				\item tabela: objeto
				\item colunas: atributos
				\item linhas: tuplas
			\end{itemize}			
			Cada tupla é considerada um elemento, o atributo representa a menor informação do objeto.\\
			Ademais, podemos ter duas visões de uma tabela: o esquema e a instância. O \textbf{esquema} representa a estrutura dos dados e seus atributos dentro da tabela. Já a \textbf{instância} representa a coleção de tuplas daquela tabela em determinado momento.
		\subsubsection{Restrições do Modelo Relacional}
			\begin{enumerate}
				\item \textbf{Domínio}: Tipo de dados e tamanho de atributo de uma tabela.
				\item \textbf{Valores Nulos}:  Permite identificar atributos como opcionais.
				\item \textbf{Chave}: Os atributos de uma tabela devem ser únicas e não podem se repetir entre as tuplas. Uma super-chave é um conjunto de atributos que caracterizam unicamente uma tupla da tabela, todos os atributos de uma tablea forma uma super-chave até o momento que, ao retirar qualquer atributo desta super-chave, esta para de identificar unicamente as tuplas. Uma chave é uma super-chave mínima.
				\item \textbf{Integridade Referencial}: Mantém a integridade das relações entre as tuplas. Uma chave estrangeira deve sempre apontar para um atributo chave.
			\end{enumerate}
		\subsubsection{Notação para representar um esquema com algumas restrições}
			\begin{itemize}
				\item \textbf{Atributo obrigatório}: sem símbolo especial.
				\item \textbf{Atributo opcional}: sublinhado com pontilhado.
				\item \textbf{Atributo chave primária}: sublinhado tradicional.
				\item \textbf{Atributo chave}: asterisco após o nome.
				\item \textbf{Atributo chave estrangeira}: nome(att(nomeTabReferenciada)
			\end{itemize}
	\subsection{Modelo Físico}

\end{document}