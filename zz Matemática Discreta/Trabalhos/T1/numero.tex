\documentclass[•]{article}
\title{Descrição aprofundada sobre o conceito de número}
\author{Erickson G. Müller}

\begin{document}
	\maketitle
	\newpage
	\section{Número}
		Números são usados há muito na nossa comunicação. Representados por símbolos, podemos usá-los para representar qualquer numeral. Com a adição de novos símbolos, podemos representar inclusive números complexos e irracionais.
		Diferente das letras, os números seguem uma regra de ordenação que deve ser aplicada para qualquer sistema de numeração com o qual estaremos trabalhando. O numeral de menor valor sempre estará à extrema direita, com o número à esquerda valendo $n$ vezes o que valeria na casa à direita. Considerando $n$ como sendo a ordem de codificação do número.
		
	\section{Sistemas de Numeração}
		Para representar os números, podemos dividir os sistemas de numeração em 2 esquemas:
		
		\subsection{Numeração Não-Posicional}
			Numeração não-posicional é aquela que o valor de um símbolo não se altera, independente da posição em que ele se encontra no conjunto de símbolos que representa um número.\\
Exemplo: Numeração Romana Antiga.

		\subsection{Numeração Posicional}
			Numeração Posicional é aquela que o valor de um símbolo depende da posição em que este se encontra no conjunto de símbolos que representam um número.\\
Exemplo: Numeração com base decimal e Binária.

	\section{Conjuntos Numéricos}
		Os números podem ser agrupados em 6 conjuntos numéricos, de acordo com suas devidas semelhanças.
		
		\subsection{Números Reais}
			Corresponde a todos os números que estão no intervalo $(-\infty,\infty)$. Dentro do conjunto dos Reais estão os Naturais, Inteiros, Racionais e Irracionais.
		\subsection{Números Racionais}
			Representa os números que podem ser descritos através de uma fração, e por serem escritos na forma de fração, também podem ser representados por um número finitos de caracteres numerais. Dentro do conjunto dos Racionais estão os Inteiros e os Naturais.
		\subsection{Números Inteiros}
			Números que são positivos ou negativos, mas que não têm parte decimal.
		\subsection{Números Naturais}
			Números positivos que não têm parte decimal.
		\subsection{Números Irracionais}
			Números que não podem ser representados por frações irredutíveis, existe a necessidade de usar uma fração ou um símbolo especial para tal número.\\
			Ex:$\pi$, $\sqrt[•]{2}$
		\subsection{Números Complexos}
			O conjunto dos complexos representa raízes de índice par e radicando negativo. Não se encaixa dentro dos Números Reais.
\end{document}