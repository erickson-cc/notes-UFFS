\documentclass{article}
\usepackage[]{amsmath}
\usepackage{pgfplots}
\pgfplotsset{compat=1.18}
\usepackage{amssymb}
\usepackage[]{pxfonts}
\usepackage[english]{babel}
\usepackage{tikz}
\usepackage[]{cancel}

\title{Revisão Prova 2 de Cálculo II\\Milton Kist}
\author{Erickson Giesel Müller}
\date{2 de Dezembro de 2024}
\begin{document}
	\maketitle
	
	\section*{Conteúdos}
		\begin{enumerate}
			\item Funções de várias variáveis (Definição, Domínio, Imagem, Operações, Representação Gráfica).
			\item Limite e Continuidade.
			\item Limite e Continuidade de funções de várias variáveis.
			\item Limites por caminho.
			\item Cálculo de Limites envolvendo indeterminações.
			\item Verificação de Continuidade de funções.
			\item Derivadas parciais e aplicações.
			\item Gradiente.
			\item Multiplicadores de Lagrange.
			\item Integração dupla.
			\item Integração tripla
			
		\end{enumerate}
	\newpage
	
	\section{Função de Várias Variáveis}
		Seja $A\subset \mathbb{R}^n$, a relação $f_i A \to \mathbb{R}$ é denominada função real, $P=(x_1,x_2,...,x_n), P\in \mathbb{R}$, associamos um único número real $z \in \mathbb{R}$.
		$$A=D(f)$$
		$$\mathbb{R}=CD(f)$$
		$$Im(f)=\{z \in \mathbb{R}/z=f(x_1,x_2,...)\}$$
		\textbf{Exemplo:} Dada a função $f(x,y)=\sqrt[]{1-x^2-y^2}$, determine os conjuntos domínio e imagem de $f$.
		$$1-x^2-y^2 \geq 0 \leftrightarrow x^2+y^2 < 1$$
		$$D(f) = \{(x,y) \in \mathbb{R}^2/x^2+y^2\leq 1\}$$
		$$Im(f)=[0,1]$$
		
		\textbf{Exemplo:} em cada caso, determine o domínio da função, faça também a representação geométrica do domínio:
	
	%aula que eu faltei abaixo
	\section{Regra da Cadeia}
		Vamos definir a regra da cadeia para o caso de funções de várias variáveis.\\
		Proposição 1: Sejam A e B conjuntos abertos do $\mathbb{R}^2$ e $\mathbb{R}$, respectivamente, e sejam $f:A \to \mathbb{R}$ e $g:B \to \mathbb{R}^2$ tais que $g(t) = (x(t),y(t)) \in A$ para todo $t \in B$. Nestas condições, se $g$ for diferenciável em $B$ e $f(x,y)$ possuir derivadas parciais de 1ª ordem contínuas em $A$.\\
		Então, a função composta:\\
			$$h(t)= f(g(t))=f(x(t),y(t))$$
		é diferenciável $\forall t \in B$ e $\dfrac{dh}{dt}$ é dada por:
			$$\dfrac{dh}{dt} = \dfrac{\varphi f}{\varphi x}. \dfrac{dx}{dt}+\dfrac{\varphi f}{\varphi y}.\dfrac{dy}{dt}$$
 \end{document}