\documentclass[ ]{article}
\usepackage[ ]{indentfirst}
\title{Revisão P1\\Sistemas Operacionais}
\author{Erickson Müller}
\date{03 de outubro}

\begin{document}
	\maketitle
	\section*{Conteúdos}
		\begin{enumerate}
			\item Processos
			\item Threads
			\item Impasses
			\item Gerenciamento de memória
		%	\item Tratamento de entrada e saída
		\end{enumerate}
	\newpage
	\section{Processos}
	
		O sistema operacional tem o modo kernel/supervisor e o modo usuário.
		
		Árvore de processos. O sistema operácional usa a hierariquia de processos para reservar a memória, carregar as partes do processo.
		
		O primeiro processo da árvore inicia o sistema operacional
		
		Os processos podem criar algum processo filho para caso necessite abrir outro programa para executar o comando.
		
		O pipe é um canal de comunicação entre dois processos.
		
		\subsection{Modelo em quatro partes}
		O espaço de endereçamento da memória de um processo é dividido em quatro partes:
			\begin{enumerate}
				\item Code/Text
				\item Data
				\item Heap/Lacuna
				\item Stack/Pilha
			\end{enumerate}
	\section{Threads}
	\section{Impasses}
	\section{Gerenciamento de Memória}
%z	\section{Entrada e Saída}
\end{document}