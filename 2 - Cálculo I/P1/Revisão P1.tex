\documentclass{article}
\usepackage[]{amsmath}
\usepackage{amssymb}
\usepackage[english]{babel}
\usepackage{tikz}
\usepackage[]{cancel}

\title{Revisão Prova 1 de Cálculo I\\Divane Marcon}
\author{Erickson Giesel Müller}
\begin{document}
	\maketitle
	
	\section{Introdução}
		Para o desenvolvimento deste resumo, foi utilizado o livro da Diva Marília Flemming \textit{ Cálculo A: funções, limite, derivação e integração.}
		Além das aulas e listas fornecidas pela professora Divane.\\
		
	\section{Funções}
		\subsection{Noção Intuitiva}
			Em um gráfico de uma função, o limite é a tendência à qual a imagem está se aproximando conforme o x se aproxima, dos dois lados, pelo número determinado no limite.\\
			Às vezes a curva passa pelo ponto em que o x está se aproximando, e esse limite é exato. Outras, como no caso quando o limite é infinito ou a função se aproxima ao eixo das abscissas, a função nunca intercepta realmente aquela imagem, mas tende a ele.
			
		\subsection{Exemplos}
			Para calcular a função abaixo, quando tende a 2 quando x = +$\infty$
			\begin{equation*}
				y = \frac{2x+1}{x-1}
			\end{equation*}
		\subsection{Unicidade do Limite}
			Se $\lim_{x\to a}$f(x) = $L_{1}$ e $\lim_{x\to a}$f(x) = $L_{2}$ então $L_{1} = L_{2}$
			\\ \\
			\[ \lim_{x\to 2}(x^2+3x+5) = \lim_{x\to 2}x^2 + 3.\lim_{x\to 2}x + \lim_{x\to 2}5 \]
		\subsection{Teorema do Confronto}
			\begin{minipage}{5 cm}
				 \[ a) \lim_{x\to 0} sen \frac{1}{x} = \nexists \]
			\end{minipage}
			\begin{minipage}{5 cm}
				\[ b) \lim_{x\to 0} xsen \frac{1}{x} = 0\]
			\end{minipage}
			\\ \\
			\begin{minipage}{5 cm}
				\[ c) \lim_{x\to 0} x^2sen \frac{1}{x} = 0\]
			\end{minipage}
			\begin{minipage}{5 cm}
				\[ d) \lim_{x\to 0} x^3sen \frac{1}{x} = 0 \]
			\end{minipage}
			\\ \\
			Por mais que $\frac{1}{0}$ seja um absurdo, ao multiplicar por um outro limite que resulte em zero, todo o conjunto resultará zero.
			\\ \\
			No teorema de sanduíche, é escolhido uma função maior igual e outra menor igual no ponto em que o limite da função f(x) será calculado.
			\[ g(x) \leq f(x) \leq h(x)\]
			\[ g(x) \leq \lim_{x\to 0}xsen \frac{1}{x} \leq h(x)\]
			Como o seno varia de 1 até menos 1, temos que:
			\[ -1. \lim_{x\to 0}x \leq \lim_{x\to 0}x . \lim_{x\to 0}sen \frac{1}{x} \leq 1.\lim_{x\to 0}x \]
			logo:
			\[ -1.0 \leq \lim_{x\to 0}xsen \frac{1}{x} \leq 1.0 \]
			\[ \lim_{x\to 0}xsen \frac{1}{x} = 0\]
			
		\subsection{Limites Laterais}
			Seja \textit{f} uma função definida no intervalo (a,c). Dizemos que um número L é o limite \textbf{à direita} da função \textit{f} quando x tende para a e escrevemos:
				\[ \lim_{x\to a^+}f(x)= L\]
			Esse limite é real se para todo $\varepsilon >0$ existe um $\delta>0$ tal que $|f(x)-L|<\varepsilon$ sempre que $a<x<a+\delta^*$.\\
			\hspace*{4cm} *
			\begin{minipage}{8cm}
				Essa função é semelhante ao teorema do sanduíche. O x se encontra entre o valor ao qual se aproxima e esse valor + $\delta$.
			\end{minipage}\\ \\
			Ou seja, no gráfico, para um limite que se aproxima pela direita, precisa necessariamente existir uma imagem para todo domínio dentro do $\delta$ à direita.\\
			\\
			 Seja \textit{f} uma função definida no intervalo (d,a). Dizemos que um número L é o limite \textbf{à esquerda} da função \textit{f} quando x tende para a e escrevemos:
				\[ \lim_{x\to a^-}f(x)= L\]
			Esse limite é real se para todo $\varepsilon >0$ existe um $\delta>0$ tal que $|f(x)-L|<\varepsilon$ sempre que $a-\delta<x<a$.\\
			\textbf{TEOREMA:} se \textit{f} é definida em um intervalo aberto contendo \textit{a}, exceto no ponto a, então $\lim_{x\to a}f(x) = L$ se $\lim_{x\to a^+}f(x) = L$ e $\lim_{x\to a^-}f(x) = L$.\\ \\
			Ou seja, só existe limite nos dois lados se existe o mesmo limite no lado direito e no lado esquerdo.
			
		\subsection{Indeterminações}
			\begin{equation*}
				\frac{0}{0}, \frac{\infty}{\infty}, \infty - \infty, 0.\infty, 0^0, \infty^0,1^\infty	
			\end{equation*}
			\begin{enumerate}
				\item Sejam $f(x) = x^3$ e $g(x) = x^2$, temos:\\
				\[ \lim_{x\to 0}f(x) = \lim_{x\to 0}g(x) = 0 \]
				e
				\[ \lim_{x\to 0}\frac{f(x)}{g(x)} = \lim_{x\to 0}\frac{x^3}{x^2} = \lim_{x\to 0}
				\frac{\cancel{x^{3}}}{\cancel{x^2}} = \lim_{x\to 0}x = 0\]
				
				\item Sejam $f(x) = x^2$ e $g(x) = 2x^2$, temos:\\
				\[ \lim_{x\to 0}f(x) = \lim_{x\to 0}g(x) = 0\]
				e
				\[ \lim_{x\to 0}\frac{f(x)}{g(x)} = \lim_{x\to 0}\frac{\cancel{x^2}}{2\cancel{x^2}} = \frac{1}{2} \]
				
				\item Seja:
				\[ \lim_{x\to 0}\frac{senx}{x}\]
				Essa função causa uma indeterminação $\frac{0}{0}$, podemos separar essa divisão em duas funções, sendo $f(x) = senx$ e $g(x) = x$. À medida que os valores de $x$ pertencem a uma pequena vizinhanças de 0, a função $f(x)$ se aproxima de 0, assim como a a função $g(x)$ também se aproxima de 0, ficando cada vez mais próximos entre si (Diferença entre $f(x)$ e $g(x)$ cada vez menor à medida que x se aproxima de 0). Podemos, portanto, assumir que:
				\[ \lim_{x\to 0} \frac{senx}{x} = \lim_{x\to 0} \frac{f(x)}{g(x)} = 1 \]
				
				\item Agora um exemplo de  cálculo de limites onde os artifícios algébricos são necessários. São os casos de funções racionais em que o limite do denominador é zero num determinado ponto e o limite do numerador também é zero neste mesmo ponto.\\
				
				Seja:\\
				\[ \lim_{x\to -2}\frac{x^3-3x+2}{x^2-4} \]
				Neste caso, fatora-se o numerador e o denominador VER PREPOSIÇÕES 3.5.2
			\end{enumerate}
			
\end{document}
