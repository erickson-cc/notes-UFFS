\documentclass{article}
\usepackage[]{amsmath}
%\usepackage[]{pxfonts}
\usepackage[english]{babel}

\title{Revisão Prova 1 de\\Organização de Computadores\\Luciano L. Caimi}
\author{Erickson Giesel Müller}
\date{}

\begin{document}
	\maketitle
	\section*{Conteúdos}
		\begin{enumerate}
			\item Arquitetura Von Neumann
			\item Arquitetura Harvard
			\item Arquitetura RISC
			\item Arquitetura CISC
			
			\item Pipeline
			\item Hierarquia de memória
				\begin{enumerate}
				 	\item Cache
				 	\item Memória Principal RAM
				 	\item Memória Virtual
				 	\item Armazenamento de Massa
				\end{enumerate}
		\end{enumerate}
		
	\section{Memória Cache}
		\subsection{Cache com Mapeamento Direto}
			Para calcular a linha da memória cache:
			$$i = j \mod m$$
			onde:\\
			$i$ é o número da linha na memória cache\\
			$j$ é o número de blocos na memória RAM\\
			$m$ é o número de blocos na memória cache\\
			Todas as relações de endereço são em potência de 2. Assim, conseguimos encontrar a linha na memória cache, mas, em uma operação de busca, ainda precisamos saber em qual \underline{bloco} está a informação, para isso usamos duas informações de controle:			
		\subsection{Bit de Válido}
			Indica se a entrada do cache está escrita com alguma informação.
		\subsection{Bit de Tag}
			Complementa o cálculo de endereço com a sintaxe $[tag]j \mod m$. Os bits de tag informam em qual parte da memória cache está a informação de acordo com a linha calculada.
\end{document} 